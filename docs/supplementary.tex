\documentclass[a4paper]{article}

\usepackage[utf8]{inputenc}
\usepackage[T1]{fontenc}
\usepackage{textcomp}
%\usepackage[dutch]{babel}
\usepackage{amsmath, amssymb}
\usepackage{graphicx}
\usepackage[pdfusetitle]{hyperref}
\graphicspath{{figures/}}
\usepackage{comment}
\usepackage{parskip}

\pdfsuppresswarningpagegroup=1


% Definitions for displaying regular expressions
\makeatletter
% \literalset\foo<SPACE>ARBITRARY CHARACTERS<END OF LINE>
\def\literalset #1{% assumes standard \endlinechar
    \begingroup
    \def\x{#1}%
    \catcode`\^^M 2
    \let\do\@makeother
    \dospecials
    \afterassignment\literalset@i
    \toks0=\bgroup }%
\def\literalset@i 
   {\expandafter\xdef\x{\expandafter\@gobble\the\toks0}\endgroup}
\makeatother

\makeatletter
\def\printliteral   #1{\expandafter\printliteral@i#1\relax }%
\def\printliteral@i #1{\if\relax #1\else\hskip\z@ \@plus .4\p@\relax
                       #1\expandafter\printliteral@i \fi}
\makeatother


\title{Supplementary Materials}
\date{September 2022}
\author{Edwin Wenink}

\begin{document}

\maketitle
\tableofcontents

\section{Regular expression for detecting case sentences}\label{sec:regex}

Albeit being natural language in free form, juridical lingo tends to use relatively standardized formulations from common situations.
This also holds the final verdict as summarized at the end of a case transcription.
It is therefore feasible to extract information about the verdict in a structured manner using regular expressions.
This appendix explains the used regular expressions, how they are processed, and discusses their limitations.

\subsection{Main punishments}

Dutch law distinguishes four main types of punishment: prison sentence, custody, community service and a fine.
These all share a common feature, namely that the judge is always required to specify the length or height of the punishment.
This is different for TBS, which does not have a pre-set duration, and for acquittal, which is a binary decision.
Rather than designing multiple regular expressions for different punishments and different situations, I designed a single regular expression to match all of them.
This implies a shift in mindset from only matching exactly the information we need (so that we can directly use all matches), to capturing all potentially relevant information and then defining a rule-based classifier on the captured information.
The latter approach avoids having to do extra repeated passes over the input text, avoids capturing the same information twice, and allows for more flexibility because the rule-based classifier is easier to adjust than the regular expression itself.
The complete pattern used for the main punishments is:

% https://regex101.com/r/6fbiBD/10
% https://regex101.com/r/6fbiBD/18
%\literalset\hoofdstrafpattern (?i)(?:(?P<modifier1>voorwaardelijk|proeftijd|niet|vervangend|indien|mindering|maatregel)[^\n\r;.]{0,100})?\b(?P<straf>gevangenis|gevangenisstraf|jeugddetentie|detentie|hechtenis|taakstraf|werkstraf|leerstraf|geldboete|vordering(?!\stenuitvoerlegging)(?!\stot\stenuitvoerlegging)|betaling)\b(?P<test1>[^\n\r;.]{0,85}?)(?P<nummer1>(?<!feit )\d+(?!\s?\]))(?P<test2>[^0-9\n\r;.]{0,30}?)(?P<eenheid1>jaar|jaren|maanden|maand|week|weken|dag|dagen|uur|uren|euro|,[-\d=]{1,2}|(?:\/|-|:)?[\d.]+(?!\s?\])(?:,[-\d=]{1,2})?)(?P<niettest1>[^0-9\n\r;.]{0,15})(?:(?P<nummer2>(?<!feit )\d+(?!\s?\]))[^0-9\n\r;.]{0,30}?(?:\s(?P<eenheid2>jaar|jaren|maanden|maand|week|weken|dagen|dag|uur|uren)))?(?:(?P<niettest2>[^\n\r;.]{0,150})(?P<modifier2>voorwaardelijk|proeftijd|niet|vervangend|indien|hechtenis|wederrechtelijk))?
\literalset\hoofdstrafpattern (?i)(?:(?P<modifier1>voorwaardelijk|proeftijd|niet|vervangend|indien|mindering|maatregel)[^\n\r;.]{0,100})?\b(?P<straf>gevangenis|gevangenisstraf|jeugddetentie|detentie|hechtenis|taakstraf|werkstraf|leerstraf|geldboete|vordering(?!\stenuitvoerlegging)(?!\stot\stenuitvoerlegging)|betaling)\b(?P<test1>[^\n\r;.]{0,85}?)(?P<nummer1>(?<!feit )\d+(?!\s?\]))(?P<test2>[^0-9\n\r;.]{0,30}?)(?P<eenheid1>jaar|jaren|maanden|maand|week|weken|dag|dagen|uur|uren|euro|,[-\d=]{1,2}|(?:\/|-|:)?[\d.]+(?!\s?\])(?:,[-\d=]{1,2})?)(?P<niettest1>[^0-9\n\r;.]{0,15})(?:(?P<nummer2>(?<!feit )\d+(?!\s?\]))[^0-9\n\r;.]{0,30}?(?:\s(?P<eenheid2>jaar|jaren|maanden|maand|week|weken|dagen|dag|uur|uren)))?(?:(?P<niettest2>[^\n\r;.]{0,150})(?P<modifier2>voorwaardelijk|proeftijd|niet|vervangend|indien|hechtenis|wederrechtelijk))?

\texttt{\printliteral\hoofdstrafpattern}

This regular expression is long and quite unreadable (as most regular expressions are), but it does have some repeated subcomponents that we will explain part by part here.
The matching is case insensitive, indicated by \texttt{(?i)}.
The most relevant components are captured as named groups, indicated with the \texttt{(?P<name>)} syntax.
This syntax is specific to the native Python \texttt{re} module.

\paragraph{Main punishment}

\literalset\hoofdstraffen \b(?P<straf>gevangenis|gevangenisstraf|jeugddetentie|detentie|hechtenis|taakstraf|werkstraf|leerstraf|geldboete|vordering(?!\stenuitvoerlegging)(?!\stot\stenuitvoerlegging)|betaling)\b

\texttt{\printliteral\hoofdstraffen}

The most important component is the indication of the type of punishment.
Some synonyms are included that are mapped to the four formal names in the downstream labelling procedure.
For example, ``werkstraf'' and ``taakstraf'' both indicate community service.
The word ``detentie'' (detention) is ambiguous and could both mean a prison sentence and custody.
We treat it as a synonym for custody (``hechtenis''), because we observe in the text data that this word is typically used for sentences with less than a year of detainment.
The word ``vordering'' often indicates a monetary claim and thus relates to a fine.
However, there are of course other types of claims.
A very common one is a claim to execute a conditional punishment of which the conditions are violated (``vordering (tot) tenuitvoerlegging straf'').
This common case is excluded by doing a negative lookahead on ``tenuitvoerlegging'' and ``tot tenuitvoerlegging.''
We include a zero-width word boundary assertion e.g. to exclude matching ``vordering'' as in: ``het Wetboek van Strafvordering ten hoogste kan worden gevorderd op 3 jaren'' (ECLI:NL:RBLIM:2020:9692).

\paragraph{Punishment height}

\literalset\nummereen (?P<test1>[^\n\r;.]{0,85}?)(?P<nummer1>(?<!feit )\d+(?!\s?\]))

\texttt{\printliteral\nummereen}

We then want to know the length or height of the punishment, which is either an amount of time (in hours, days, months, or years) or money (in euros).
Monetary sums are always written out as digits.
The length of the other sentences are almost without exception also indicated with digits in the following canonical form: 
``veroordeelt de verdachte tot een gevangenisstraf voor de duur van 2 (twee) jaar.''
Of course, digits may occur for a variety of reasons, so we need to ensure that a matching digit indicates the appropriate measurement.
This is done in several ways. 

First of all, the digit should be related to the main type of punishment and therefore we require the digit to occur in a particular window after the indication of the punishment.
We allow a break of maximum 85 characters. 
This window size has been empirically determined by evaluating test cases.
Moreover, we require the digit to occur in the same sentence and break the match when a new sentence is detected (I refer to the regex parts such as \texttt{[\string^\string\n\string\r;.]} as ``connectors'').
Semicolons are heavily used in enumerations in the case transcriptions and are also treated as a line separator.
This approach makes all regular expressions also work when all newlines are stripped from the case text.

Secondly, the matched digits are always coupled with the detection of the unit of measurement (see below for more details).
If we find a prison sentence but the amount is expressed in euros, then we know the matched digits do not express the length of the prison sentence and we reject the match in the downstream logic.
Likewise, if we find a fine but the digit expresses an amount of time, we reject the match.
These situation occur because when a fine is imposed, there is typically an expression of a replacement punishment in case the fine is not paid.
Examples are: ``hechtenis heeft doorgebracht naar rato van 50 euro per dag'' and ``betaling en verhaal te vervangen door 100 dagen gijzeling.''
The specification of such a replacement punishment is more or less a formality and the height of the replacement punishment is coupled to the height of the original punishment.
Since we are interested in detecting the main sentence, we can safely discard these matches.

Thirdly, we have to deal with some edge cases.
All case transcriptions are anonymized, as follows: 
``Wijst de vordering van de benadeelde partij [slachtoffer 1] toe tot een bedrag van  € 1.314,28 (duizend driehonderdveertien euro en achtentwintig cent).''
The 1 in ``[slachtoffer 1]'' prevents a match of the amount of euros.
Because this is a very common scenario that occurs in practically each case transcription, we do a negative lookahead and only match digits that are not succeeded by the closing bracket `]'.
In a sentence like ``betaling aan de benadeelde partij [Slachtoffer 3] (feit 9) van € 5.226,53'' we also want to exclude the number 9 from the fine.
This is done with a negative lookbehind for a preceding ``feit''.
Digits also frequently occur in dates and in case identifiers (e.g. the ``parketnummer'').
See below how they are handled.

\paragraph{Unit of measurement}
\literalset\eenheideen (?P<test2>[^0-9\n\r;.]{0,30}?)(?P<eenheid1>jaar|jaren|maanden|maand|week|weken|dag|dagen|uur|uren|euro|,[-\d=]{1,2}|(?:\/|-|:)?[\d.]+(?!\s?\])(?:,[-\d=]{1,2})?)

\texttt{\printliteral\eenheideen}

Once we know the amount, we need to know the unit of measurement (hours, days, months, years, or euros).
Again, we allow a limited window of characters between the number and the unit, e.g. as in ``duur van 2 (twee) jaar''.
The temporal units are self-explanatory, but there is more variety in how an amount of euros is displayed, e.g.: ``2000 euro'', ``€ 87,66'', ``€ 87'', ``87,--'', ``87,='', ``€ 2.662,63''.
A complication is that our regular expression requires the unit corresponding to the matched digit to come \emph{after} the digit.
This is only explicit in the case of ``x \emph{euro}.''
In the other cases, we need to infer that a number refers to an amount of euros.
Because euro signs appear in front of the number, they are not matched as a unit (a positive lookback does not suffice in this case, because the euro sign is already consumed by the connector regex).
In cases where the matching digit is succeeded e.g. by ``,66'' or ``,--'', then these suffixes indicate that we are dealing with money and are matched as the unit.
We apply similar reasoning for higher fines such as ``€ 2.662,63'', but in this case ``.662,63'' is matched as the suffix.
These cases can be easily handled in the downstream logic by checking if the unit starts with a comma or period and if so, reconstruct the total fine by appending the suffix (with some additional preprocessing) to the matched digit.

This approach catches all scenarios except ``€ 20'', in which case ``0'' is matched as the unit.
If the unit consists only of digits like in this case and occurs after an indication for a fine, then we parse it as a fine, unless the digit occurs in a context that we filter out as an exception, such as when it indicates a numbered fact (``feit'') or a date.
%In this case appending ``0'' to ``2'' happens to be a valid result in this example, but in ``vordert betaling voor feit 10 begaan op 10 januari 2020'' the same logic will result in a non-existent fine of 1010 euros.
%This issue is addressed in two ways.
%1)
%We capture the connector preceding the first digit and check for the presence of a euro sign.
%If other euro indicators are absent, we only append digits if a euro sign is present before the first matching digit.
%2)
%We only record fines if they occur behind an indicator for a fine, such as ``geldboete'', and discard them otherwise.
%Similarly, we know that a community service has to be specified either in terms of hours or days since the maximum punishment is 10 days / 240 hours.
%In this case we clamp the height to the maximum value and raise a warning, so that the data point may be discarded later if desired.
%We furthermore catch the most common patterns that we do not want to match.
These exceptions include dates (e.g. ``27-01-2021'') and numeric identifiers (``parketnummer 23/003276-17'' or ``artikel 6:6:25 van Wetboek van Strafvordering'').
These are handled downstream by discarding matches with a unit starting with ``-'' or ``/`` or ``:''.
Note that we could have opted for extending the negative lookahead (currently only for ``feit'') behind the first digit.
This would however only move the problem because e.g. in ``27-01-2021'' this would match ``2'' as the number and ``7'' as the unit. 
Similarly, in ``2-01-2021'' this would skip the first ``2'', but then match the next ``2'' and the ``1'' as the unit.
The benefit of the chosen approach is that we can now consistently recognize if we are dealing with an identifier or date downstream, because the symbols such as ``:'' are included in the match.
Another check is to control that no month names occur between matching digits, such as in ``betaling van dit bedrag, vermeerderd met de wettelijke rente over dit bedrag vanaf 7 mei 2016 tot aan'' (ECLI:NL:RBNHO:2020:11590).
%However, strictly speaking this check is superfluous because this match is also discarded due to the absence of a euro sign.
%We also do a negative lookahead on closing square brackets to avoid matching ``6'' as a unit in ``vordering van de benadeelde partij [benadeelde partij 16] tot een bedrag van 150 euro'' (ECLI:NL:RBAMS:2021:7066).

Because we limited the scope to main punishments (with the exception of TBS), we do not want to match another type of payment, such as the measure to return unlawfully obtained advantages (goods or money).
We perform additional checks for the occurrence of ``wederrechtelijk verkregen voordeel'' (unlawfully obtained advantages), ``schadevergoeding'' or ``smartegeld'' (compensation), as well as checking for ``maatregel'' (measure), using the ``test'' and ``modifier'' named capture groups.

When a fine is given, the sum of money is assigned to the person(s) making the claim (``vordering'').
However, in Dutch law the sum is not directly paid to the claimant, but instead to the state which, in turn, pays out the sum to the claimant.
This means that strictly speaking there are two money transactions for a single fine.
This is also linguistically represented in case decisions: first we'll see that the fine is awarded to the claimant, but secondly we see an explicit statement that the defendant is obliged to pay that same sum to the state.
We want to match the first case, but not the second in order to avoid doubling the amount of fines.
Many of the duplicate instances have a phrasing that is not matched by the regular expression (e.g. ``Legt verdachte de verplichting op ten behoeve van [naam slachtoffer] aan de Staat € 5.000,- (vijfduizend euro) te betalen''), but this does not always hold.
We do another check on presence of ``aan de Staat'' to avoid matches such as: ``bepaalt dat verdachte verplicht is ter zake van het bewezen verklaarde feit tot betaling aan de Staat der Nederlanden van een bedrag van € 436,27''.

\paragraph{Sentences with components}

\literalset\deeltwee(?:(?P<nummer2>\d+(?!\s?\]))[^0-9\n\r;.]{0,30}?(?:\s(?P<eenheid2>jaar|jaren|maanden|maand|week|weken|dagen|dag|uur|uren)))?
\texttt{\printliteral\deeltwee}

In particular for punishments such as prison sentences, the sentence is typically written out explicitly as such:  ``gevangenisstraf van 12 (twaalf) maanden en 6 (zes) maanden''.
We repeat a similar procedure as described above for the additional sentence component, the main difference being that we only accept temporal units and that the second component is an optional match. 

\paragraph{Modifiers}

\literalset\modifiereen(?:(?P<modifier1>voorwaardelijk|proeftijd|niet|vervangend|indien|mindering|maatregel)[^\n\r;.]{0,100})?
\texttt{\printliteral\modifiereen}

\literalset\modifiertwee(?P<modifier2>voorwaardelijk|proeftijd|niet|vervangend|indien|hechtenis|wederrechtelijk))?
\texttt{\printliteral\modifiertwee}

Finally, we match several keywords, which we call ``modifiers,'' that are important for the interpretation of the match.
These modifiers are optionally matched either at the beginning or the end of the regular expression.
It is possible that a particular punishment is mentioned precisely because it will \emph{not} be imposed.
The following examples are some negations from the data set that are detected such that the mentioned punishments are discarded:
``verklaart de benadeelde partij niet ontvankelijk in de vordering ter hoogte van € 1.176,50 voor de kosten van de huishoudelijke hulp'' (ECLI:NL:RBROT:2020:11499);
``detentie groot 30 (dertig) dagen, niet ten uitvoer zal worden gelegd, tenzij de rechter later anders mocht gelasten wegens niet'' (ECLI:NL:RBROT:2020:12683);
``bij niet betaling te vervangen door 204 dagen gijzeling, met dien verstande dat toepassing van de gijzeling de betalingsverplichting niet opheft'' (ECLI:NL:RBZWB:2020:6268).

A limitation of the regex is that these negations are only matched at the beginning or end of a match.
We therefore do an additional check on the keywords ``niet'', ``niet ten uit'' and ``niet tenuit'' in the connector groups called ``niettest1'' and ``niettest2,'' to catch edge cases where the negation occurs somewhere else.
This is also necessary due to the connector group between ``eenheid1'' and ``nummer2'' being greedy: this often consumes ``niet'' because the ending modifier group is optional.
Making this connector group lazy introduces a nasty side-effect that prevents matching the optional later parts of the regex (such as ``en 6 (zes) maanden''): it will first look for matches without expanding; which succeeds because the rest of the string can be matched by the ``niettest2'' group.
There is no ``incentive'' to backtrack and look for ``nummer2'' instead.
Due to this extra check, the following case is correctly recognized as \emph{not} being executed:
''bepaalt dat van deze gevangenisstraf een gedeelte, groot 10 (tien) weken, niet ten uitvoer zal worden gelegd, tenzij de rechter later anders mocht gelasten;'' (ECLI:NL:RBROT:2020:13002).

The keywords ``voorwaardelijk'' (conditional) and ``proeftijd'' (probation) indicate that (a part of) a sentence is conditional.
This is a more tricky case, because in some cases we arguably want to include a conditional punishment and in others not.
In the following example, a conditional prison sentence is part of the punishment:
``gevangenisstraf van 2 weken voorwaardelijk met een proeftijd (...)'' (ECLI:NL:RBZWB:2020:6449).
However, in the following cases we do \emph{not} want to add the conditional part of the sentence: ``gevangenisstraf van 12 (twaalf) maanden, waarvan 6 (zes) maanden voorwaardelijk met een proeftijd'' (ECLI:NL:RBZWB:2020:6775); 
 ``gevangenisstraf van 18 (achttien) maanden, waarvan 3 (drie) maanden voorwaardelijk met een proeftijd (...)'' (Case: ECLI:NL:RBZWB:2020:6166).
This is a very common phrasing.
If the optional second part of a sentence has the modifier ``voorwaardelijk'' or ``proeftijd'' we therefore exclude this part from the total sentence sum.

In addition to specifying conditional punishments or a conditional part to an otherwise non-conditional punishment, it is also very common to specify a replacement punishment in case some other requirement is not fulfilled.
The keywords ``indien'' (if) and ``vervangend'' (replacing) and ``subsidiair'' (subsidiary) indicate that a sentence is a subsidiary punishment in case the main punishment is not executed properly.
We are interested in the original sentence and skip these matches.
The matching punishments in the following example is discarded:
``indien verdachte de taakstraf niet naar behoren verricht, vervangende hechtenis zal worden toegepast van 30 dagen'' (ECLI:NL:RBZWB:2020:6449; N.B. in this example ``hechtenis'' is correctly matched as the punishment, instead of ``taakstraf'').
The subsidiary punishment is typically a form of custody (``hechtenis'') in case a community service or a fine has not been paid.
In the following case it is not sufficient to check for ``vervangend'' in the ending modifier:
``taakstraf van 80 (tachtig) uren, met bevel dat indien deze straf niet naar behoren wordt verricht vervangende hechtenis zal worden toegepast voor de duur van 40 (veertig) dagen''.
In this case we correctly recognize a community service of 80 hours in ``taakstraf van 80 (tachtig) uren, met bevel dat indien deze straf niet naar behoren wordt verricht vervangend'', but then incorrectly start a new match at ``hechtenis'' so that the subsidiary custody is counted as a punishment of its own: ``hechtenis zal worden toegepast voor de duur van 40 (veertig) dagen''.
This case is handled by also matching ``hechtenis'' as an ending modifier if it occurs in reasonably close vicinity to a previously mentioned punishment, which is appropriate because it is so commonly a replacement punishment.
This will then capture ``taakstraf van 80 (tachtig) uren, met bevel dat indien deze straf niet naar behoren wordt verricht vervangende hechtenis'' and avoids a further match starting with the subsidiary custody.
In some instances a subsidiary punishment is mentioned in the past tense, as in: ``hechtenis heeft doorgebracht naar rato van 50 euro per dag''.
Detection of ``[heeft|is] doorgebracht'' will lead to rejection of the match.
This works well enough because legal experts tend to use the exact same phrasings over and over, but conceptually speaking an improvement would be to apply a parser to detect the past tense in matches.

It is also very common that a suspect has already been detained while awaiting the case to appear in court.
The judge's verdict will, for legal reasons, explicitly mention that this custody will be detracted from the total punishment, if applicable.
In order to avoid \emph{adding} a punishment that is in fact \emph{detracted} we check for the phrase ``in mindering'', either by matching it as the first modifier or in the ``test1'' capture group.
We additionally test for the presence of ``heeft doorgebracht'' or ``is doorgebracht'' in the ``test1'' group.
This avoids incorrectly detecting a community service for example in the following sentence: ``bepaalt dat de tijd die verdachte voor de tenuitvoerlegging van deze uitspraak in voorarrest heeft doorgebracht in mindering wordt gebracht bij de tenuitvoerlegging van de taakstraf naar rato van 2 uur per dag''.

%Matching ``maatregel'' in ``modifier1'' and ``wederrechtelijk'' (as in ``wederrechtelijk verkregen voordeel'') additionally helps filter out sums of money that are measures instead of fines for crimes. 

Finally, we have domain knowledge on the maximum height of sentences (30 years for prison sentence, 1 year for custody, and 10 days for community service).
If we find a punishment that exceeds the legally allowed maximum, this is certainly due to a parsing error.
We detect these cases we clip the height to the maximum value and raise a warning, so that these data points may later be filtered out if desired.

\subsection{TBS}

% \literalset\tbs (?i)(?P<verlenging>verlengt|verlenging)[^\n\r;.]{0,50}(?P<TBS1>TBS|terbeschikkingstelling|ter beschikking (?:stelling|gesteld))|(?P<TBS2>TBS|terbeschikkingstelling|ter beschikking (?:stelling|(?:\w+\s)?gesteld))[^\n\r;.]{0,50}(?P<type>voorwaarden|verpleging)'

\literalset\tbs (?i)(?:(?P<verlenging>verlengt|verlenging).{0,50})?(?P<TBS>TBS|terbeschikkingstelling|ter beschikking (?:wordt |is )?(?:stelling|gesteld))(?:(?!voorwaarde|verple).){0,100}(?P<type>voorwaarden|verpleging|verpleegd)?

\texttt{\printliteral\tbs}

A notable difference with the main punishments is that phrases imposing TBS do not explicitly specify a duration, because TBS is imposed for 2 years and then re-evaluated and possibly extended.
There are two types of TBS: either TBS is imposed with mandatory nursing (``dwangverpleging'' of ``verpleging van overheidswege'') or with a set of conditions (``met voorwaarden'') such as a stay in a forensic psychiatric hospital or taking certain medications. 
The regex therefore requires either an indication for mandatory nursing or an indication of conditions, next to the indication for TBS.
However, there are also often cases where it is decided to prolong a TBS measure that has previously been imposed.
In these cases there is often an indication of the duration, but this is typically until the next review moment, which is a standardized moment and hence not interesting to parse.

The phrase ``ter beschikking stellen'' is heavily used in legal jargon in different contexts as well, so in order to avoid false positives we do require either an indication of prolongation of a previous TBS measure, or an indication of the type.
We may either enforce this in the regular expression itself or handle this downstream in the rule-based classifier.
If enforcing this in the regular expression itself, the expression would have the logical form:

$(verlenging \land TBS) \lor (TBS \land type)$

This duplicates the TBS expression and on average will require the regex engine to take significantly more steps for each match.
We instead opted for an expression with an optional ``verlenging'' group before the TBS expression and an optional ``type'' group after it, i.e. we capture all possible relevant information and discard matches when they do not meet our requirements.
In this case, the rule-based classifier discards a match when neither of the optional groups is matched.

The following examples are all detected:
``gelast de terbeschikkingstelling van verdachte, met verpleging van overheidswege;'' (ECLI:NL:RBZWB:2020:6268);
``gelast dat de verdachte, voor de feiten 2, 3 en 4, ter beschikking wordt gesteld en stelt daarbij de volgende, het gedrag van de ter beschikking gestelde betreffende, voorwaarden'' (ECLI:NL:RBLIM:2020:9778);
``De rechtbank verlengt de termijn van de terbeschikkingstelling van veroordeelde met één jaar.'' (ECLI:NL:RBNNE:2020:4558);
``verlengt de termijn gedurende welke [verdachte] ter beschikking is gesteld met verpleging van overheidswege met één jaar'' (ECLI:NL:RBLIM:2020:10468).


\subsection{Acquittal}

% Old version
%\literalset\vrijspraak (?i)(?P<vrijspraak>vrijgesproken|vrijspraak|spreekt[^.\r\n;]*\svrij|wijst[^\r\n;]{0,100}\saf)(?:(?!meer of anders).){0,50}(?P<nebisinidem>meer of anders (?:ten laste is|is ten laste) gelegd)?

\literalset\vrijspraak (?i)((?P<nebisinidem1>meer of anders (?:ten laste is gelegd|is ten laste gelegd|is tenlastegelegd|tenlastegelegd is)).{0,50})?(?P<vrijspraak>vrijgesproken|vrijspraak|spreekt[^.\r\n;]*\svrij|wijst[^\r\n;]{0,100}\saf)(?:(?!meer of anders).){0,50}(?P<nebisinidem2>meer of anders (?:ten laste is gelegd|is ten laste gelegd|is tenlastegelegd|tenlastegelegd is))?

\texttt{\printliteral\vrijspraak}

This regular expression is more straightforward.
Consider the following representative examples:
``spreekt de verdachte vrij'' (ECLI:NL:RBLIM:2020:9690); 
``wijst de vordering van de benadeelde partij voor het overige af'' (ECLI:NL:RBLIM:2020:9690); 
``wijst de vordering tot immateriële schade voor het overige af'' (ECLI:NL:RBGEL:2020:6988).

The phrase ``wijst voor het overige af'' (rejection of remaining claims) is a construct that can be used to summarize acquittal on several claims.
In many cases we find a similar phrase which, however, has a different special significance.
For example, in case ECLI:NL:RBZWB:2020:6800 we find:

\begin{quote}
- verklaart het ten laste gelegde bewezen, zodanig als hierboven onder 4.4 is omschreven;

- spreekt verdachte vrij van wat meer of anders is ten laste gelegd; 
\end{quote}

This indicates acquittal of all points except the proven facts mentioned in section 4.4.
The facts in section 4.4 are, however, the only facts mentioned in the case.
This phrasing rather indicates that there will be no repeated prosecution on closely related facts.
This is called the \emph{ne bis in idem} (``not twice in the same'') principle in civil law; its equivalent being the ``double jeopardy'' clause in common law. 
This principle states that one cannot be persecuted twice for the same facts, although there may be dispute about which courses of actions constitute new facts and which do not.
This principle applies when a fact has already been judged by a foreign judge or a Dutch judge, or when a settlement is agreed upon and the suspect has paid.
A fact is considered to be judged when at least one material question (as opposed to the formal questions, such as the determination whether the fact is punishable) has been answered by the judge.

In any case, the relevance for the labelling of case outcomes is that even though this phrase linguistically resembles acquittal, it is equally applicable in situations where a punishment is imposed as in those where the suspect is acquitted of all charges.
This means that we should recognize this situation and not label it as ``acquittal''.
This is done with the following expression:

% \literalset\nebisinidem (?:(?!meer of anders).){0,50}(?P<nebisinidem>meer of anders (?:ten laste is|is ten laste) gelegd)?
\literalset\nebisinidem (?:(?!meer of anders).){0,50}(?P<nebisinidem2>meer of anders (?:ten laste is gelegd|is ten laste gelegd|is tenlastegelegd|tenlastegelegd is))?
\texttt{\printliteral\nebisinidem}

Because the ``ne bis in idem'' clause is optional, it is necessary to temper the scope of the \texttt{.} token with a negative lookahead.
This avoid that the construct \texttt{.{0,50}} expands over the optional group and never matches it, even if it is present.
This construct is also used in the regular expression for TBS.
The phrase indicating ``ne bis in idem'' can also occur before the indication of acquittal, as in ``verklaart niet bewezen hetgeen aan de verdachte meer of anders ten laste is gelegd dan hiervoor bewezen is verklaard en spreekt de verdachte daarvan vrij'', which we match with an optional beginning group.
This beginning group is a literal match that anchors the pattern, so we do not need extra tricks for tempering the scope of \texttt{.{0,50}} here.

We find that if we do not exclude the formal ``ne bis in idem'', acquittal is detected in the vast majority of cases, which gives an incorrect picture.

 
\subsection{Test cases}\label{sec:strafmaat_test_cases}

To provide insight into the behavior of the regular expressions and the rule-based classifier defined on the matches, we provide test cases that represent several scenarios.
A full suite of tests can be found in the code repository.

%NIEUWE TEST CASE? wijst de vordering tot materiële schade voor een gedeelte van € 280,- af; (ECLI:NL:RBGEL:2021:1003)

% wijst de vordering van de officier van justitie tot tenuitvoerlegging van de bij vonnis van de rechtbank van 13 november 2018 voorwaardelijk opgelegde jeugddetentie van 3 maanden af (ECLI:NL:RBGEL:2021:1003)

\paragraph{Simple cases}

The following cases represent typical situations.
The duration of a punishment is practically always mentioned with a digit, alongside the number fully written out.
The duration is almost without exception mentioned \emph{after} a mention of the type of punishment.
Punishments are converted to days or euros, except for ``TBS'' and ``vrijspraak'' which are binary.

\texttt{TEST CASE: %0
een gevangenisstraf van 5 (vijf) jaren\\
MATCH GEVANGENISSTRAF: gevangenisstraf van 5 (vijf) jaren\\
OUT: \{'TBS': 0, 'gevangenisstraf': 1825, 'hechtenis': 0, 'taakstraf': 0, 'geldboete': 0, 'vrijspraak': 0\}}

\texttt{TEST CASE: %1
veroordeelt de verdachte tot hechtenis voor de duur van 3 (drie) maanden;\\
MATCH HECHTENIS: hechtenis voor de duur van 3 (drie) maanden\\
OUT: \{'TBS': 0, 'gevangenisstraf': 0, 'hechtenis': 91, 'taakstraf': 0, 'geldboete': 0, 'vrijspraak': 0\}}

Community service is often expressed in terms of hours.
In this case we round up to an integer amount of days.

\texttt{TEST CASE: %32:
Persoon wordt veroordeeld tot taakstraf van 40 (veertig) uur\\
MATCH TAAKSTRAF: taakstraf van 40 (veertig) uur\\
OUT: \{'TBS': 0, 'gevangenisstraf': 0, 'hechtenis': 0, 'taakstraf': 2, 'geldboete': 0, 'vrijspraak': 0\}}

\paragraph{Compounded durations}

In the case of prison sentences, the duration is often compounded.
A duration of 2.5 year is written as ``2 (two) years and 6 (six) months'', so requires two matches of the time unit and conversion of both units into days in order to sum them.

\texttt{TEST CASE: % 3:
veroordeelt de verdachte tot een gevangenisstraf voor de duur van 2 (twee) jaar en 6 maanden;\\
MATCH GEVANGENISSTRAF: gevangenisstraf voor de duur van 2 (twee) jaar en 6 maanden\\
OUT: \{'TBS': 0, 'gevangenisstraf': 912, 'hechtenis': 0, 'taakstraf': 0, 'geldboete': 0, 'vrijspraak': 0\}}

\paragraph{Conditions, probation, subsidiary punishments}

However, in several instances we find two units of time that we do not want to sum up together.
A very similar situation is where a conditional part of an unconditional punishment is specified.
We should not add the conditional part to the sentence length.

\texttt{TEST CASE: % 14:
veroordeelt verdachte tot een gevangenisstraf van 12 (twaalf) maanden, waarvan  6 (zes) maanden voorwaardelijk met een proeftijd van twee jaar;\\
MATCH GEVANGENISSTRAF: gevangenisstraf van 12 (twaalf) maanden, waarvan  6 (zes) maanden voorwaardelijk met een proeftijd\\
Conditional punishment detected.\\
WARNING: Second part of sentence is conditional. This part is excluded.\\
OUT: \{'TBS': 0, 'gevangenisstraf': 365, 'hechtenis': 0, 'taakstraf': 0, 'geldboete': 0, 'vrijspraak': 0\}}

We often have a situation where the probation period of a sentence is explicitly mentioned.

\texttt{TEST CASE: % 6:
voorwaardelijke gevangenisstraf van 5 jaar, en verbindt hieraan een proeftijd, die wordt gesteld op 2 jaar;\\
MATCH GEVANGENISSTRAF: voorwaardelijke gevangenisstraf van 5 jaar, en verbindt hieraan een proeftijd\\
Conditional punishment detected.\\
OUT: \{'TBS': 0, 'gevangenisstraf': 1825, 'hechtenis': 0, 'taakstraf': 0, 'geldboete': 0, 'vrijspraak': \}}

For fines and community service we often find a replacement or subsidiary punishment in case the first one is not fulfilled.
When this is not recognized this can lead to very wrong results, such as community service in the order of 30 or 50 days, whereas the legal maximum is 10 days.

\texttt{TEST CASE: % 34:
taakstraf bestaande uit het verrichten van onbetaalde arbeid voor de duur van 60 (zestig) uren, subsidiair 30 dagen hechtenis\\
MATCH TAAKSTRAF: taakstraf bestaande uit het verrichten van onbetaalde arbeid voor de duur van 60 (zestig) uren, subsidiair 30 dagen hechtenis\\
Subsidiary punishment detected. This part is excluded.\\
OUT: \{'TBS': 0, 'gevangenisstraf': 0, 'hechtenis': 0, 'taakstraf': 3, 'geldboete': 0, 'vrijspraak': 0\}}

\texttt{TEST CASE: % 35:
wanneer taakstraf niet naar behoren heeft verricht, wordt vervangende hechtenis toegepast van 50 (vijftig) dagen\\
MATCH TAAKSTRAF: taakstraf niet naar behoren heeft verricht, wordt vervangende hechtenis toegepast van 50 (vijftig) dagen\\
Subsidiary punishment detected. Skipped.\\
OUT: \{'TBS': 0, 'gevangenisstraf': 0, 'hechtenis': 0, 'taakstraf': 0, 'geldboete': 0, 'vrijspraak': 0\}}

\paragraph{Reduction}

Often a suspect will have spent time in custody and this time will be reduced from whatever sentence is ultimately imposed.
We want to keep the original sentence that is imposed for the crime, but not add (nor subtract) the time already spent in custody.

\texttt{TEST CASE: % 41:
bepaalt dat de tijd die verdachte voor de tenuitvoerlegging van deze uitspraak in voorarrest heeft doorgebracht in mindering wordt gebracht bij de tenuitvoerlegging van de taakstraf naar rato van 2 uur per dag;\\
MATCH TAAKSTRAF: mindering wordt gebracht bij de tenuitvoerlegging van de taakstraf naar rato van 2 uur per dag\\
'in mindering' detected. Skipped.\\
OUT: \{'TBS': 0, 'gevangenisstraf': 0, 'hechtenis': 0, 'taakstraf': 0, 'geldboete': 0, 'vrijspraak': 0\}}

In some cases the judge will mention time already spent in custody in past tense without the key words ``in mindering''.
We also filter these out.

\texttt{TEST CASE: % 42:
hechtenis heeft doorgebracht naar rato van 50 euro per dag\\
MATCH HECHTENIS: hechtenis heeft doorgebracht naar rato van 50 euro per dag\\
Previous punishment detected. Skipped.\\
OUT: \{'TBS': 0, 'gevangenisstraf': 0, 'hechtenis': 0, 'taakstraf': 0, 'geldboete': 0, 'vrijspraak': 0\}}

\paragraph{Negations}

It occurs regularly that a (part of a) sentence is mentioned precisely because it is \emph{not} executed.
These negations should be detected.

\texttt{TEST CASE: % 9:
gevangenisstraf een gedeelte van 120 (honderdtwintig) dagen niet ten uitvoer\\
MATCH GEVANGENISSTRAF: gevangenisstraf een gedeelte van 120 (honderdtwintig) dagen niet ten uit\\
Negation ('niet ten uitvoer') detected. Skipped.\\
OUT: \{'TBS': 0, 'gevangenisstraf': 0, 'hechtenis': 0, 'taakstraf': 0, 'geldboete': 0, 'vrijspraak': 0\}}

\texttt{TEST CASE: % 15:
van deze gevangenisstraf zal een gedeelte, groot 3 (drie) maanden, van deze gevangenisstraf niet tenuitvoergelegd worden, tenzij later anders wordt gelast. Stelt daarbij een proeftijd van 2 (twee) jaren vast.\\
MATCH GEVANGENISSTRAF: gevangenisstraf zal een gedeelte, groot 3 (drie) maanden, van deze gevangenisstraf niet\\
Negation detected at beginning or end of match. Skipped.\\
OUT: \{'TBS': 0, 'gevangenisstraf': 0, 'hechtenis': 0, 'taakstraf': 0, 'geldboete': 0, 'vrijspraak': 0\}}

\paragraph{Fines}

Fines may be written in a variety of ways, e.g. as `\texteuro 5.000,-', `\texteuro 60', `500,= euros' and so on.
We discard cents.

\texttt{TEST CASE: % 18:
Wijst de vordering van de benadeelde partij [naam slachtoffer] toe tot een bedrag van € 5.000,- (vijfduizend euro) aan vergoeding van immateriële schade\\
MATCH VORDERING: vordering van de benadeelde partij [naam slachtoffer] toe tot een bedrag van € 5.000,- (vijfduizend e\\
OUT: \{'TBS': 0, 'gevangenisstraf': 0, 'hechtenis': 0, 'taakstraf': 0, 'geldboete': 5000, 'vrijspraak': 0\}}

Often, the components of the total fine are also mentioned. We want to avoid matching and summing the subcomponents as well, otherwise we will double the height of the fine.

\texttt{TEST CASE: % 27:
Wijst de vordering van de benadeelde partij [slachtoffer 1] toe tot een bedrag van  € 1.314,28 (duizend driehonderdveertien euro en achtentwintig cent), bestaande uit € 314,28 (driehonderdveertien euro en achtentwintig cent) aan vergoeding van materiële schade en € 1.000,00 (duizend euro) aan vergoeding van immateriële schade, te vermeerderen met de wettelijke rente daarover vanaf het moment van het ontstaan van de schade op 27 juni 2020 tot aan de dag van de algehele voldoening.\\
MATCH VORDERING: vordering van de benadeelde partij [slachtoffer 1] toe tot een bedrag van  € 1.314,28 (duizend drieh\\
OUT: \{'TBS': 0, 'gevangenisstraf': 0, 'hechtenis': 0, 'taakstraf': 0, 'geldboete': 1314, 'vrijspraak': 0\}}

When a fine is imposed on the suspect, the suspect has to pay the fine to the state on behalf of the person making the claim.
Often this obligation to pay the state with the same amount is mentioned separately, which again is a risk for doubling the height of the fine.
This is an important reason for not just using a more simple regular expression for matching all amounts of money, but only matching amounts of euros in certain conditions.
The imposed fine is typically preceded by the term ``vordering'' (claim), whereas the payment to the state that we do not want to match is preceded by another term such as ``verplichting'' (obligation).

\texttt{TEST CASE: % 31:
Wijst de vordering van de benadeelde partij [naam slachtoffer] toe tot een bedrag van € 5.000,- (vijfduizend euro) aan vergoeding van immateriële schade. Legt verdachte de verplichting op ten behoeve van [naam slachtoffer] aan de Staat € 5.000,- (vijfduizend euro) te betalen.\\
MATCH VORDERING: vordering van de benadeelde partij [naam slachtoffer] toe tot een bedrag van € 5.000,- (vijfduizend e\\
OUT: \{'TBS': 0, 'gevangenisstraf': 0, 'hechtenis': 0, 'taakstraf': 0, 'geldboete': 5000, 'vrijspraak': 0\}}

% TODO waar noem ik maatregelen?

Sometimes it is explicitly mentioned that we are dealing with a measure instead of a punishment.
If the ``modifier1'' group matches ``maatregel'' we discard the match.
If ``maatregel'' is absent but we see a phrase indicating payment to the State (``aan de Staat''), then we are also almost certain the mentioned sum is duplicated and discard the match.

\texttt{TEST CASE: % 58:
legt de maatregel op dat verdachte verplicht is ter zake van het bewezen verklaarde feit tot betaling aan de Staat der Nederlanden van een bedrag van € 436,27, te vermeerder\\
MATCH BETALING: maatregel op dat verdachte verplicht is ter zake van het bewezen verklaarde feit tot betaling aan de Staat der Nederlanden van een bedrag van € 436,27, te vermeerder\\
Measure ('maatregel') detected. Skipped\\
OUT: \{'TBS': 0, 'gevangenisstraf': 0, 'hechtenis': 0, 'taakstraf': 0, 'geldboete': 0, 'vrijspraak': 0\}}

\texttt{TEST CASE: %59:
dat verdachte verplicht is ter zake van het bewezen verklaarde feit tot betaling aan de Staat der Nederlanden van een bedrag van € 436,27, te vermeerderderen\\
MATCH BETALING: betaling aan de Staat der Nederlanden van een bedrag van € 436,27, te vermeerder\\
Payment to state detected. Probably duplicated sum. Skipped.\\
OUT: \{'TBS': 0, 'gevangenisstraf': 0, 'hechtenis': 0, 'taakstraf': 0, 'geldboete': 0, 'vrijspraak': 0\}}

\paragraph{TBS}

TBS (``terbeschikkingstelling'') is imposed with conditions (``voorwaarden''), mandatory nursing (e.g. ``verpleging van overheidswege''), or prolongation (``verlenging'').

\texttt{TEST CASE: % 48:
gelast dat de verdachte, voor de feiten 2, 3 en 4, ter beschikking wordt gesteld en stelt daarbij de volgende, het gedrag van de ter beschikking gestelde betreffende, voorwaarden (ECLI:NL:RBLIM:2020:9778)\\
MATCH TBS: ter beschikking wordt gesteld en stelt daarbij de volgende, het gedrag van de ter beschikking gestelde betreffende, voorwaarden\\
OUT: \{'TBS': 1, 'gevangenisstraf': 0, 'hechtenis': 0, 'taakstraf': 0, 'geldboete': 0, 'vrijspraak': 0\}}

\texttt{TEST CASE: % 49:
verlengt de termijn gedurende welke [verdachte] ter beschikking is gesteld met verpleging van overheidswege met één jaar" (ECLI:NL:RBLIM:2020:10468)\\
MATCH TBS: verlengt de termijn gedurende welke [verdachte] ter beschikking is gesteld met verpleging\\
OUT: \{'TBS': 1, 'gevangenisstraf': 0, 'hechtenis': 0, 'taakstraf': 0, 'geldboete': 0, 'vrijspraak': 0\}}

``Ter beschikking stellen'' is used in other legal contexts as well, so we do not match this term by itself if it occurs without conditions, mandatory nursing, or prolongation.

\texttt{TEST CASE: % 51:
ter beschikking stelling van de goederen aan benadeelde partij\\
MATCH TBS: ter beschikking stelling van de goederen aan benadeelde partij\\
WARNING: neither 'verlenging' nor 'type' of TBS detected. Skipped.\\
OUT: \{'TBS': 0, 'gevangenisstraf': 0, 'hechtenis': 0, 'taakstraf': 0, 'geldboete': 0, 'vrijspraak': 0\}}

\texttt{TEST CASE: % 52:
TBS kliniek De Kijvelanden, wederrechtelijk van de vrijheid heeft beroofd en/of beroofd gehouden, immer\\
MATCH TBS: TBS kliniek De Kijvelanden, wederrechtelijk van de vrijheid heeft beroofd en/of beroofd gehouden, immer\\
WARNING: neither 'verlenging' nor 'type' of TBS detected. Skipped.\\
OUT: \{'TBS': 0, 'gevangenisstraf': 0, 'hechtenis': 0, 'taakstraf': 0, 'geldboete': 0, 'vrijspraak': 0\}}

\paragraph{Acquittal}

Acquittal on some fact is relatively easy to match.

\texttt{TEST CASE: % 43:
spreekt de verdachte daarvan vrij\\
MATCH VRIJSPRAAK: spreekt de verdachte daarvan vrij\\
OUT: \{'TBS': 0, 'gevangenisstraf': 0, 'hechtenis': 0, 'taakstraf': 0, 'geldboete': 0, 'vrijspraak': 1\}}

\texttt{TEST CASE: % 44:
wijst de vordering van de benadeelde partij voor het overige af\\
MATCH VRIJSPRAAK: wijst de vordering van de benadeelde partij voor het overige af\\
OUT: \{'TBS': 0, 'gevangenisstraf': 0, 'hechtenis': 0, 'taakstraf': 0, 'geldboete': 0, 'vrijspraak': 1\}}

In some cases we find an indication of acquittal that is a formal statement that excludes further prosecution on the same facts.
The following case finds the suspect guilty on all charges but contains a formal phrase that indicates this ``ne bis in idem'' principle (cf. ``no double jeopardy'').

\texttt{TEST CASE: % 45:
- verklaart het ten laste gelegde bewezen, zodanig als hierboven onder 4.4 is omschreven; - spreekt verdachte vrij van wat meer of anders is ten laste gelegd;\\
MATCH VRIJSPRAAK: spreekt verdachte vrij van wat meer of anders is ten laste gelegd\\
'ne bis in idem' detected. Skipped.\\
OUT: \{'TBS': 0, 'gevangenisstraf': 0, 'hechtenis': 0, 'taakstraf': 0, 'geldboete': 0, 'vrijspraak': 0\}}

\paragraph{No measures}

In particular amounts of money are easily confused with measures as opposed to punishments.
Because we have limited our scope to main punishments, we filter out measures where possible for the sake of consistency.

\texttt{TEST CASE: % 55:
legt [verdachte] de verplichting op tot betaling aan de staat ter ontneming van het wederrechtelijk verkregen voordeel van € 331.083,14 (zegge: driehonderdeenendertigduizend drieëntachtig euro en veertien eurocent)\\
MATCH BETALING: betaling aan de staat ter ontneming van het wederrechtelijk verkregen voordeel van € 331.083,14 (zegge: drieho\\
Measure to return unlawfully obtained advantages detected. Skipped.\\
OUT: \{'TBS': 0, 'gevangenisstraf': 0, 'hechtenis': 0, 'taakstraf': 0, 'geldboete': 0, 'vrijspraak': 0\}}

\texttt{TEST CASE: % 56:
veroordeelt verdachte in verband met het feit onder nummer 1 en 2 tot betaling van schadevergoeding aan de benadeelde partij [getuige 1] van  37,48 aan materiële schade en  1.500,- aan smartengeld, vermeerderd met de wettelijke rente vanaf 22 november 2019 tot aan de dag dat het hele bedrag is betaald\\
MATCH BETALING: betaling van schadevergoeding aan de benadeelde partij [getuige 1] van  37,48 aan materiële
Measure for compensation detected. Skipped.\\
OUT: \{'TBS': 0, 'gevangenisstraf': 0, 'hechtenis': 0, 'taakstraf': 0, 'geldboete': 0, 'vrijspraak': 0\}}

\paragraph{Edge cases}

We avoid matching dates and case identifiers as fines.

 % 65:
\texttt{TEST CASE:
vordering van de officier van justitie tot tenuitvoerlegging in de zaak met parketnummer 23/003276-17\\
MATCH VORDERING: vordering van de officier van justitie tot tenuitvoerlegging in de zaak met parketnummer 23/003276-\\
Identifier detected, e.g. a date, case number, law reference. Skipped.\\
OUT: \{'TBS': 0, 'gevangenisstraf': 0, 'hechtenis': 0, 'taakstraf': 0, 'geldboete': 0, 'vrijspraak': 0\}}

 % 67:
\texttt{TEST CASE:
veroordeelt verdachte tot betaling van het toegewezen bedrag voor de schade ontstaan op 4 juli 2020\\
MATCH BETALING: betaling van het toegewezen bedrag voor de schade ontstaan op 4 juli 2020\\
WARNING: Date detected with month juli. Skipped\\
OUT: \{'TBS': 0, 'gevangenisstraf': 0, 'hechtenis': 0, 'taakstraf': 0, 'geldboete': 0, 'vrijspraak': 0\}}

We often find digits that enumerate certain charges or facts and that should not be matched as the height of a punishment.
These are skipped with a negative lookbehind.

 % 25:
\texttt{TEST CASE: betaling aan de benadeelde partij [Slachtoffer 3] (feit 9) van € 5.226,53,\\
MATCH BETALING: betaling aan de benadeelde partij [Slachtoffer 3] (feit 9) van € 5.226,53,\\
OUT: \{'TBS': 0, 'gevangenisstraf': 0, 'hechtenis': 0, 'taakstraf': 0, 'geldboete': 5226, 'vrijspraak': 0\}}

%ECLI: ECLI:NL:RBAMS:2021:7066
%MATCH TAAKSTRAF: taakstraf van 180 (hondertachtig) uren
%MATCH HECHTENIS: niet naar behoren heeft verricht, dat vervangende hechtenis zal worden toegepast van 90 (negentig) dagen
%Negation detected at beginning or end of match. Skipped.
%MATCH TAAKSTRAF: taakstraf niet ten uitvoer zal worden gelegd, tenzij verdachte zich voor het einde van de op 2 (twee) jaren gestelde proef
%WARNING: community service has to be specified in hours or days. Skipped!

\subsection{Failure analysis}\label{sec:failure_analysis}

Regular expressions are surprisingly effective for extracting the sentences from case decisions because juridical language use is relatively structured.
There are however some exceptions that the regular expressions do not capture.
The good f1-score of 0.94 shows that the following situations are indeed exceptions.
It is possible to extend the used regular expressions and the rule-based classifier, but within the scope of this project a labelling score of 0.94 is more than sufficient.

The used regular expression assumes that the name of the type of punishment precedes the duration of the punishment.
Cases where this order is reversed are not matched or wrongly matched in some edge cases:

 % 2:
\texttt{TEST CASE:
Veroordeelt verdachte tot honderdtachtig (180) dagen jeugddetentie.\\
OUT: {'TBS': 0, 'gevangenisstraf': 0, 'hechtenis': 0, 'taakstraf': 0, 'geldboete': 0, 'vrijspraak': 0}\\
NO MATCHES FOUND
}

\texttt{
TEST CASE: % 43:
Gelast de tenuitvoerlegging van de werkstraf, voor zover voorwaardelijk opgelegd bij vonnis van de kinderrechter van Rechtbank Noord-Nederland, locatie Leeuwarden van 6 oktober 2020, te weten: 50 uren werkstraf subsidiair 25 dagen vervangende jeugddetentie\\
MATCH WERKSTRAF: werkstraf subsidiair 25 dagen vervangende\\
OUT: {'TBS': 0, 'gevangenisstraf': 0, 'hechtenis': 0, 'taakstraf': 2, 'geldboete': 0, 'vrijspraak': 0}\\
}

We also assume the length of the punishment is indicated with a digit, but because the anonymization scheme of rechtspraak.nl uses digits within brackets (e.g. `[feit 1]' or `[persoon 1]') we avoid matching digits followed by a square closing bracket.
In very rare instances, we may see stylistic inconsistencies that prevent matching.
%For example, square brackets are used for indicating anonymized persons (e.g. ``[benadeelde 1]'').
I have found at least one case where use of squares instead of regular brackets for the sentence length avoided a match.
This match is rejected because the ``nummer1'' regex group matches ``2'' and the ``eenheid1'' regex group matched ``4'', which normally would indicate we are dealing with an amount of euros.
The rule-based classifier then complains that we find a prison sentence with a fine, which is inconsistent.

 % 67:
\texttt{TEST CASE: gevangenisstraf voor de duur van vierentwintig [24] maanden
MATCH GEVANGENISSTRAF: gevangenisstraf voor de duur van vierentwintig [24] maanden\\
Amount of euros found, but punishment is not a fine\\
OUT: \{'TBS': 0, 'gevangenisstraf': 0, 'hechtenis': 0, 'taakstraf': 0, 'geldboete': 0, 'vrijspraak': 0\}}


In some cases no digit is mentioned at all.
Recognizing numbers written out in natural language would require a parser; regular expressions are not a good solution for this.
Luckily, this situation occurs rarely.

 % 32:
\texttt{TEST CASE:
veroordeelt verdachte tot een taakstraf van honderdtwintig uren;\\
OUT: {'TBS': 0, 'gevangenisstraf': 0, 'hechtenis': 0, 'taakstraf': 0, 'geldboete': 0, 'vrijspraak': 0}\\
NO MATCHES FOUND
}

Typically, we correctly match the total amount of money without adding up its components, but if the components are mentioned first after the indication of punishment we incorrectly match the subcomponent as the total:

% 28:
\texttt{TEST CASE: vordering van de benadeelde partij [persoon] toe tot een bedrag van € 87,-- (zevenentachtig euro) aan vergoeding van materiële schade en € 1.000 aan immateriele. \\
MATCH VORDERING: vordering van de benadeelde partij [persoon] toe tot een bedrag van € 87,-- (zevenentachti\\
OUT: {'TBS': 0, 'gevangenisstraf': 0, 'hechtenis': 0, 'taakstraf': 0, 'geldboete': 87, 'vrijspraak': 0}\\
}

% FIXED?
% TEST CASE 25: betaling aan de benadeelde partij [Slachtoffer 3] (feit 9) van € 5.226,53,
% MATCH BETALING: betaling aan de benadeelde partij [Slachtoffer 3] (feit 9) van € 5.226,53,
% WARNING: Indication for fine detected, but (9,5.226,53) is no valid amount of euros
% OUT: {'TBS': 0, 'gevangenisstraf': 0, 'hechtenis': 0, 'taakstraf': 0, 'geldboete': 0, 'vrijspraak': 0}
% NO MATCHES FOUND
% PASSING: False
 
 
 
 
 
 
 
 
 
 
 
 
% NOTITIES FAILURES / MEESTE AL GEFIXT
\begin{comment}

Deze is niet te ondervangen; afwijkende parketnummer indicatie; punt komt ook in geldige nummers voor.

ECLI: ECLI:NL:RBOVE:2021:1983
MATCH BETALING: betaling aan de benadeelde partij [aangever 2] (parketnummer 08.215446-

``Gelast de tenuitvoerlegging van de werkstraf, voor zover voorwaardelijk opgelegd bij vonnis van de kinderrechter van Rechtbank Noord-Nederland, locatie Leeuwarden van 6 oktober 2020, te weten: 50 uren werkstraf subsidiair 25 dagen vervangende jeugddetentie.''

Zie:
ECLI: ECLI:NL:RBNNE:2021:3376
MATCH WERKSTRAF: werkstraf subsidiair 25 dagen vervangende
XXXXXXXXXXXXXXXXXXXXX
ERROR: maximum community service exceeded. There is probably a parsing mistake
XXXXXXXXXXXXXXXXXXXXX
OUT: {'TBS': 0, 'gevangenisstraf': 0, 'hechtenis': 0, 'taakstraf': 25, 'geldboete': 0, 'vrijspraak': 0}

SOLVED: "mindering" opgeteld. -> lijdt tot 11 dagen taakstraf.
---------------------------------------------
Case: ECLI:NL:RBGEL:2021:1200
---------------------------------------------
MATCH WERKSTRAF: werkstraf voor de duur van 240 uren, met bevel dat indien deze straf niet naar behoren wordt verricht vervangende hechtenis
MATCH WERKSTRAF: werkstraf in verzekering is doorgebracht, bij de uitvoering van die straf 6 uren in mindering w
MATCH VRIJSPRAAK: spreekt verdachte daarvan vrij;   verstaat dat het aldus bewezenverklaarde oplev
XXXXXXXXXXXXXXXXXXXXX
ERROR: maximum community service exceeded. There is probably a parsing mistake
OUT: {'TBS': 0, 'gevangenisstraf': 0, 'hechtenis': 0, 'taakstraf': 11, 'geldboete': 0, 'vrijspraak': 1}

Geen hechtenis van een jaar, maar een gevangenisstraf; lastige formulering though.
``Gelast met inachtneming van het in voormeld vonnis bepaalde omtrent de aftrek van de tijd door de veroordeelde in verzekering en/of voorlopige hechtenis doorgebracht, de tenuitvoerlegging van 12 maanden gevangenisstraf, waarvan, bij vonnis van deze rechtbank d.d. 1 maart 2019 tegen de veroordeelde gewezen, bevel was gegeven dat deze voorwaardelijk niet zou worden ten uitvoer gelegd.'' (ECLI:NL:RBNNE:2021:754)

SOLVED door te checken of er niet meer dan drie punten tussen de digits zitten.
``vordering van de benadeelde partij [benadeelde 6] (parketnummer 03.155784.19 feit'' (ECLI:NL:RBLIM:2020:9868) omdat een gebruikelijke use case is dat eenheid1 met een punt begint.
Wat wel zo is, is dat als dit een goed getal is, we verwachten dat er maar 3 digits tot de volgende punt zijn.

Problem: hechtenis is al doorgebracht
---------------------------------------------
Case: ECLI:NL:RBNHO:2020:10486
---------------------------------------------
MATCH HECHTENIS: hechtenis heeft doorgebracht, te weten honderdzestien (116) dagen, bij de tenuitvoerlegging van het onvoorwaardelijk
Voorwaardelijke straf detected.
MATCH BETALING: betaling van dit bedrag vermeerderd met de wettelijke rente over dit bedrag vanaf 17 juni 2020 tot aan de dag
WARNING: Date detected with month december. Skipped
MATCH BETALING: betaling aan de Staat van een bedrag van € 5.585,18, (bestaande ui
MATCH VRIJSPRAAK: spreekt hem daarvan vrij
MATCH VRIJSPRAAK: Wijst af
OUT: {'TBS': 0, 'gevangenisstraf': 0, 'hechtenis': 116, 'taakstraf': 0, 'geldboete': 5585, 'vrijspraak': 1}

Fixed (check "heeft doorgebracht" in test1): 
TEST CASE 63: hechtenis heeft doorgebracht, te weten vierenveertig (44) dagen, bij de tenuitvoerlegging van het onvoorwaardelijk
MATCH HECHTENIS: hechtenis heeft doorgebracht, te weten vierenveertig (44) dagen, bij de tenuitvoerlegging van het onvoorwaardelijk
Previous punishment detected. Skipped.
OUT: {'TBS': 0, 'gevangenisstraf': 0, 'hechtenis': 0, 'taakstraf': 0, 'geldboete': 0, 'vrijspraak': 0}
================
NO MATCHES FOUND
================
(0, 0, 0, 0, 0, 0)
PASSING: True

SOLVED with ``subsidiair'' check in niettest1 en niettest2: taakstraf van 33 dagen ``taakstraf bestaande uit het verrichten van onbetaalde arbeid voor de duur van 60 (zestig) uren, subsidiair 30 dagen'' (ECLI:NL:RBNHO:2020:10732)

SOLVED met "vervangende" check in test1: Case: ECLI:NL:RBAMS:2020:6339
MATCH TAAKSTRAF: taakstraf van 100 (honderd) uur
MATCH TAAKSTRAF: taakstraf niet naar behoren heeft verricht, wordt vervangende hechtenis toegepast van 50 (vijftig) dagen

SOLVED Probleem in ECLI:NL:RBGEL:2020:6694; taakstraf niet gematched vanwege "niet" als modifier2; nieuwe match mogelijk vanaf "hechtenis" terwijl dat de subsidiary straf is. Opgelost door "hechtenis" als modifier2 te matchen. Te verantwoorden en nummer 2 dan niet mee te tellen (als die er is). Hechtenis is heel vaak subsidiare straf:
``veroordeelt verdachte wegens de bewezenverklaarde feiten 1, 3 en 4 tot een taakstraf van 240 uren, met bevel dat indien deze straf niet naar behoren wordt verricht vervangende hechtenis zal worden toegepast voor de duur van 120 dagen;''

ibid in case: ECLI:NL:RBGEL:2020:6592:
MATCH TAAKSTRAF: taakstraf van 80 (tachtig) uren, met bevel dat indien deze straf niet naar behoren wordt verricht vervangend
MATCH HECHTENIS: hechtenis zal worden toegepast voor de duur van 40 (veertig) dagen
MATCH VRIJSPRAAK: spreekt verdachte daarvan vrij
OUT: {'TBS': 0, 'gevangenisstraf': 0, 'hechtenis': 40, 'taakstraf': 4, 'geldboete': 0, 'vrijspraak': 1}

SOLVED "Niet ten uitvoer" match naar nummer2 verplaatst en uitgebreid van "niet" naar "niet ten uitvoer". Hechtenis nu ook als modifier2 gematcht. Case: ECLI:NL:RBAMS:2020:7007. Probleem: ``niet naar behoren'' verwijdert hoofdstraf; de vervangende straf wordt juist wel genoemd.
MATCH TAAKSTRAF: taakstraf van 240 (tweehonderdveertig) uren, met bevel, voor het geval dat de verdachte de taakstraf niet naar behoren heeft verricht, dat vervangend
Negation ('niet ten uitvoer') detected
MATCH HECHTENIS: hechtenis zal worden toegepast van 120 (honderdtwintig) dagen

TODO: regex developed on data from specific time frame; evaluate on test set from different time frame.

Problem here is not just the regex; but also that the section labels are wrong. 

 ``Verlengt de in het vonnis van de meervoudige kamer van de Rechtbank Noord-Nederland,  locatie Groningen van 4 april 2019 vastgestelde proeftijd met één jaar. Dit vonnis is gewezen door mr. S. Zwarts, voorzitter, mr. J.V. Nolta en mr. T.M.L. Veen, rechters, bijgestaan door mr. S. Fokkert, griffier, en uitgesproken ter openbare terechtzitting van deze rechtbank op 1 december 2020. Mr. S. Zwarts is buiten staat dit vonnis mede te ondertekenen.'' (ECLI:NL:RBNNE:2020:4172)
 
 ibid. for ECLI:NL:RBNNE:2020:4171
 
 ibid for ECLI:NL:RBNHO:2020:11591

Sometimes there are other sections labelled as beslissing.
E.g. in case ECLI:NL:RBNHO:2020:10959, kopje called: ``Beslissing omtrent in beslag genomen, niet teruggegeven goederen''
In this case there is another beslissing section with the proper punishments.

Ibid. ECLI:NL:RBNHO:2020:10314

Typically these do not pose issues by only regarding the latest beslissing, unless the section labelling is wrong.

Cases where only maatregelen are present (not detected):

ontneming wederrechtelijk verkregen voordeel; ECLI:NL:RBNHO:2020:10112; ECLI:NL:RBNHO:2020:10109; 



Voorwaardelijke straf niet opgemerkt want
`` veroordeelt verdachte wegens de bewezenverklaarde feiten 1, 3 en 4 tot een gevangenisstraf voor de duur van 6 (zes) maanden; bepaalt dat deze gevangenisstraf niet ten uitvoer zal worden gelegd, tenzij de rechter later anders mocht gelasten omdat verdachte zich voor het einde van de proeftijd van drie (3) jaren schuldig heeft gemaakt aan een strafbaar feit;'' (ECLI:NL:RBGEL:2020:6694)


Move to discussion section?

 Degenerate example in ECLI:NL:RBNHO:2020:11168: 
``gevangenisstraf voor de duur van vierentwintig [24] maanden''
Here 4 is not matched as nummer1 due to negative lookahead on ]; therefore nummer2 is 1, eenheid1 4; this will trigger a key error.

``bepaalt de duur van de gijzeling die met toepassing van artikel 6:6:25 van het Wetboek van Strafvordering ten hoogste kan worden gevorderd op 3 jaren.'' (https://uitspraken.rechtspraak.nl/inziendocument?id=ECLI:NL:RBLIM:2020:9692)
-> gijzeling is not a hoofdstraf and hence not recognized; but then "vordering 3 jaren" is matched (from ``Strafvordering''). I should have included a word boundary somewhere -> DONE, now before straf.

TODO artikel 6:6:25 van Wetboek van Strafvordering; ECLI also uses colons.
TODO did I forget lookahead for |?

Betaling -> Also catches sometimes "wederrechtelijk verkregen voordeel'' even though I'm only intending to match "geldboete".

``tot betaling aan de staat ter ontneming van het wederrechtelijk verkregen voordeel van een bedrag van € 331.083,14 (zegge: driehonderdeenendertigduizend drieëntachtig euro en veertien eurocent);'' (ECLI:NL:RBLIM:2020:9692) 
NOt matched only because distance between betaling and money is more than 85 chars.
Ibid "problem" in ECLI:NL:RBGEL:2020:7075.
ECLI:NL:RBGEL:2020:6512.
Well it's in this case good because we are not matching that maatregel.
TODO explicitly filter out those matches.

TODO extend

Dot may block matching:
``wijst het verzoek tot wraking van mr. H.H. Dethmers af.'' (ECLI:NL:RBLIM:2020:10610)
(Regex adjust to be more liberal, but only in case of vrijspraak).
This situation doesn't occur often, because names are anonymized.
In this case, this is a wrakingsverzoek for a judge, hence the name is public.

 
\subsection{notities}

Nothing detected in case: ECLI:NL:RBOVE:2020:4493 -> TBS CASE!
``De beslissing De rechtbank: verlengt de terbeschikkingstelling van [betrokkene] met twee jaren;'' ECLI:NL:RBOVE:2020:4493


# TODO: vordering van de officier van justitie tot tenuitvoerlegging in de zaak met parketnummer 23/003276-17 en gelast de tenuitvoerlegging van de niet ten uitvoer gelegde gevangenisstraf voor de duur

interessant geval: MATCH HECHTENIS: hechtenis heeft doorgebracht naar rato van 50 euro per dag 


vordering van  wat is hier de eenheid? EDGE CASE

an € 

kent vordering van de benadeelde partij [persoon 1] en [persoon ] toe tot een bedrag van .


vordering van 

MATCH BETALING: betaling aan de benadeelde partij [Slachtoffer 3] (feit 9) van € 5.226,53, waarvan
----------------------------------------------------------------------------------------------
WARNING: Indication for geldboete but (9,5.226,53) is no valid amount of euros


TODO:

``veroordeelt de verdachte tot een gevangenisstraf voor de duur van 2 (twee) jaar en 6 (zes) maanden.''


Connector: agnostic to whether whole text is on single line or contains many newlines. Considers line endings like . , ; carriage return etc.

\literalset\modifier (?P<modifier1>voorwaardelijk|proeftijd|niet|vervangend|indien)


TODO explanation of regex for strafmaat labelling.

TODO Verschil boete vs maatregel tot terugbetalen wederrechtelijk voordeel

TODO TBS



FInes rounded down to whole euros (i.e. cents are discarded).

TODO "maatregel" as rest klasse als er geen hoofdstraf maar ook geen vrijspraak is?

TODO regex for vrijspraak

Dutch crimininal law knows four main types of punishment ('hoofdstraffen'):

    - gevangenisstraf (levenslang, tijdelijk)
    - hechtenis (but 'voorlopige hechtenis' is not a punishment! opgelegd voor overtredingen; maximum duur van een jaar)
    - taakstraf (werkstraf of leerstraf); wordt niet opgelegd als er al minstens een half jaar gevangenisstraf is opgelegd
    - geldboete volgens categorieën systeem, gaat altijd gepaard met vervangende hechtenis als aan het bedrag niet wordt voldaan, met max een dag per 25 euro en een totale max. van een jaar:
        * 1. 390 euro
        * 2. 3.900 euro
        * 3. 7.800 euro
        * 4. 19.500 euro
        * 5. 78.000 euro
        * 6. 780.000 euro

And additional punishments:

    - ontzetting van rechten (e.g. wegnemen kiesrecht, ontnemen recht op te treden als advocaat, niet mogen dien in leger)
    - verbeurdverklaring (dader verliest eigendom van voorwerpen die een rol bij het strafbare feit speelden of erdoor zijn verkregen)
    - openbaarmaking uitspraak (uitspraak wordt niet zoals normaal anoniem gepubliceerd, maar met naam; aan de schandpaal genageld)

And maatregelen:

    - plaatsing psychiatrisch ziekenhuis
    - TBS (Ter Beschikking Stelling)
    - Ontrekking aan het verkeer (van gevaarlijke objecten; lijkt op verbeurdverklaring maar kan ook worden toegepast na vrijspraak)
    - Ontneming wederrechtelijk voordeel ('Plukze' wetgeving: wegnemen van verdiensten verkregen door criminele activiteiten)
    - Schadevergoeding aan slachtoffer (dit wordt niet gezien als straf, maar als compensatie voor het slachtoffer)
    - ISD (Inrichting Stelselmatige Daders): plaatsing in een inrichting voor stelselmatige daders (cf. psych. ziekenhuis en TBS zijn voor geestesgestoorden)
    

``Gelast voorts dat de verdachte \textbf{ter beschikking wordt gesteld} en beveelt dat hij van overheidswege zal worden \textbf{verpleegd}.''

Future work: proeftijd parsen is voor ons doel niet nodig; heeft typisch een andere expressie (vaak in een bijzin; nummer na hoofdstraf) wat ons goed uitkomt, want wij matchen die niet).

vordering is ambigu, not only used for fines; also for 'vordering tenuitvoerlegging" van (previous) voorwaardelijke straf.


``gelast de tenuitvoerlegging van de bij vonnis d.d. 26 juli 2019 voorwaardelijk aan de verdachte opgelegde straf, te weten een gevangenisstraf voor de duur van twee weken.'' (ECLI:NL:RBZWB:2020:6311)   (geen numeriek)

LOGICA:

"Veroordeelt verdachte tot een gevangenisstraf voor de duur van 12 (twaalf) maanden. Beveelt dat de tijd die door veroordeelde voor de tenuitvoerlegging van deze uitspraak in verzekering en in voorlopige hechtenis is doorgebracht, bij de tenuitvoerlegging van die straf in mindering gebracht zal worden. Beveelt dat een gedeelte, groot 3 (drie) maanden, van deze gevangenisstraf niet tenuitvoergelegd zal worden, tenzij later anders wordt gelast." (ECLI:NL:RBAMS:2020:5945)

Test hypothese: als er voorwaardeijk wordt gematched, dan willen we num 2 niet meetellen
gevangenisstraf van 2 weken voorwaardelijk met een proeftijd van 2 jaar

``gevangenisstraf van 12 (twaalf) maanden, waarvan  6 (zes) maanden voorwaardelijk met een proeftijd''

Problematische frase e.g. in ECLI:NL:RBZWB:2020:6800: "- spreekt verdachte vrij van wat meer of anders is ten laste gelegd;".
Dit is gelabeled als vrijspraak, maar er wordt een plaatsing in inrichting toegekend van 2 jaar. 

Ik lees hier een zaak van iemand die een T-shirt heeft gestolen. In de beslissing staat dan:
"
- verklaart het ten laste gelegde bewezen, zodanig als hierboven onder 4.4 is omschreven;
- spreekt verdachte vrij van wat meer of anders is ten laste gelegd;
"

De tenlastelegging is alleen m.b.t. het t-shirt. Dus het vreemde is hier eigenlijk dat er hier een "vrijspraak" wordt aangeduidt van feiten die helemaal niet in de zaak worden benoemd. Wat betreft de tenlastelegging wordt de verdachte gewoon schuldig bevonden en 2 jaar in een inrichting geplaatst, dus het is niet bepaald "vrijspraak".

Is deze "vrijspraak" een formaliteit om bijv. basically te zeggen: verdachte wordt gestraft voor dit feit en daarmee is de kous af, ofzo? Waarom is die extra formulering nodig?

Deze proeftijd wordt niet opgepikt!

'''
 - veroordeelt verdachte tot een gevangenisstraf van 2 weken voorwaardelijk met een proeftijd van 2 jaar;
'''
\end{comment}



\end{document}
